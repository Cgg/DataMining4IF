\vfil
\pagebreak
\section{Classification}

Nous avons décidé de comparer les pays fournis pour
les classer en trois catégories: les pays proches des pays
membres de l'UE, les pays proches des candidats à l'entrée
dans l'UE et les pays proches des pays non candidats non
membres de l'UE.

Nous avons pris comme critère les nouvelles technologies, à savoir
le taux d'utilisateurs d'Internet et le taux d'abonnés mobiles; ainsi
que le GDP/population.
Ces critères ont permis de bien différencier les pays lors de la
créations de {\sl clusters}, nous avons donc considéré qu'il
sont de bons critères.

\subsection{{\sl Workflow}}
Voir figure \ref{wfc}

\begin{sidewaysfigure}[!h]
\begin{center}
    \caption{{\sl Workflow} classification}
    \includegraphics[width=23cm]{\PIXPATH/workflow_classif}
\label{wfc}
\end{center}
\end{sidewaysfigure}

%\FloatBarrier

\subsection{Résultat}

\begin{center}
%\rowcolors{1}{white}{gray}
\begin{longtable}{|c|c|c|}

\hline
{\bf Membre}&{\bf Candidat}&{\bf Non membre, non candidat}\\
\endhead
\hline
Antigua and Barbuda&Algeria&Angola\\
\hline
Argentina&Azerbaijan&Bangladesh\\
\hline
Australia&Belize&Benin\\
\hline
Bahamas, The&Bolivia&Bhutan\\
\hline
Brazil&Botswana&Burkina Faso\\
\hline
Chile&Congo, Rep.&Burundi\\
\hline
China&Cote d'Ivoire&Cambodia\\
\hline
Colombia&Ecuador&Cameroon\\
\hline
Costa Rica&Egypt, Arab Rep.&Canada\\
\hline
Dominican Republic&El Salvador&Cape Verde\\
\hline
Hong Kong, China&Equatorial Guinea&Central African Republic\\
\hline
Iran, Islamic Rep.&Fiji&Chad\\
\hline
Israel&Gabon&Comoros\\
\hline
Jamaica&Gambia, The&Congo, Dem. Rep.\\
\hline
Jordan&Georgia&Eritrea\\
\hline
Kuwait&Ghana&Ethiopia\\
\hline
Lebanon&Honduras&Guinea\\
\hline
Macao, China&Indonesia&Guinea-Bissau\\
\hline
Malaysia&Kazakhstan&Haiti\\
\hline
Maldives&Kyrgyz Republic&India\\
\hline
Mauritius&Libya&Japan\\
\hline
Mexico&Mauritania&Kenya\\
\hline
Morocco&Mongolia&Kiribati\\
\hline
New Zealand&Namibia&Korea, Rep.\\
\hline
Panama&Pakistan&Lao PDR\\
\hline
Peru&Paraguay&Lesotho\\
\hline
Russian Federation&Philippines&Liberia\\
\hline
San Marino&Samoa&Madagascar\\
\hline
Saudi Arabia&South Africa&Malawi\\
\hline
Seychelles&Sri Lanka&Mali\\
\hline
Singapore&Swaziland&Marshall Islands\\
\hline
St. Vincent and the Grenadines&Tajikistan&Micronesia, Fed. Sts.\\
\hline
Syrian Arab Republic&Tonga&Mozambique\\
\hline
Thailand&&Nepal\\
\hline
Trinidad and Tobago&&Niger\\
\hline
Tunisia&&Nigeria\\
\hline
Uruguay&&Papua New Guinea\\
\hline
Venezuela, RB&&Rwanda\\
\hline
Vietnam&&Sao Tome and Principe\\
\hline
&&Senegal\\
\hline
&&Sierra Leone\\
\hline
&&Solomon Islands\\
\hline
&&Sudan\\
\hline
&&Tanzania\\
\hline
&&Togo\\
\hline
&&Turkmenistan\\
\hline
&&Uganda\\
\hline
&&United States\\
\hline
&&Uzbekistan\\
\hline
&&Vanuatu\\
\hline
&&Zambia\\
\hline
\end{longtable}
\end{center}


\subsubsection{Conclusion}

On s'aperçoit que cette classification n'est pas forcément
la plus pertinente: en effet, même si des pays considérés comme
aussi développés que les pays de l'UE comme les États-Unis, le Canada
ou encore l'Australie ou le Japon sont présents; certains n'ont à première vue pas
tout à fait la même carrure que les pays de l'UE: le Panama, la Jamaïque
ou l'Uruguay ont peut-être moins leur place que ceux précédement cités. 

Cependant, ces effets de bord ne semblent pas atteindre les autres
catégories.
