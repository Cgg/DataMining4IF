\section{Introduction}

Ce rapport présente le travail que nous avons effectué au cours du projet
de fouille de données en 4IF.


\subsection{Rappel du projet}

Le projet de fouille de donnée consiste à extraire des informations d'un
set de données contenant un ensemble de 48 attributs portant sur 209 pays,
mesurés en 2007. Il faut choisir soigneusement le ou les axes d'études en
restreignant l'ensemble des pays et/ou des attributs sur lesquels la
fouille porte.\\
L'étude est ensuite conduite en utilisant l'outil Open-Source Knime,
additionné du plugin Weka.


\subsection{Axes d'études retenus}

\subsubsection{Ensemble de pays}
Afin de disposer d'un maximum d'attributs bien renseignés, et pour ne pas
se "perdre" au cours de l'étude, il nous faut diminuer le nombre de pays
étudiés.\\
Nous avons choisi de restreindre le champ de cette étude aux pays européen.
Ce choix nous permet de disposer d'un maximum d'informations et d'une
meilleure connaissance "apriori" du champ étudié (en plus d'informations
additionnelles plus facilement accessibles).\\
Plus précisément, on distingue les pays membres de l'Union Européenne et
les pays externes, candidats (Turquie, par exemple) ou non (notamment la
Suisse ou la Norvège ; les cité-états telles que Andorre ne sont pas
étudiées car aucune ne fournit suffisamment de données).\\

\subsubsection{Problématiques explorées}
En étudiant ce set de données, on souhaite explorer les questions suivantes
:
\begin{enumerate}
\item L'UE est-elle un ensemble de pays homogène ou bien peut-on déceler
des disparités entre les pays ?
\item Comment se situent les pays externes à l'UE par rapport à cette
dernière ?
\end{enumerate}

\subsubsection{Ensemble d'attributs}
Nous étudierons les problématiques ci-dessus selon plusieurs axes. Selon
l'axe on sera amené à ne conserver qu'un petit sous-ensemble d'attributs
qui seront détaillés à chaque fois.


\subsection{Méthodologie employée}

Notre méthodologie s'inspire de la méthode CRISP-DM (CRoss Industry
Standard Process for Data Mining). Elle comprend les points suivants :

\begin{enumerate}
\item Compréhension du métier : déterminer les objectifs et critères de
réussite du projet en fonction du métier étudié.
\item Compréhension des données
\item Préparation des données
\item Modélisation
    \begin{enumerate}
    \item Clustering
    \item Classification 
    \item Prédiction
    \item Association
    \end{enumerate}
\item Évaluation des résultats produits et réitération si nécessaire
\item Déploiement : ici, rédaction du compte-rendu de projet.
\end{enumerate}

Au cours de ce projet, on n'effectuera les étapes 4.3 et 4.4.
Dans ce rapport, on détaillera \textbf{les phases 3 et 4} de la méthode. 


\subsection{Glossaire}

On présente les abréviations et termes de jargons utilisés dans ce rapport
:

\begin{description}
\item[FdD,DM : ] Respectivement Fouille de Donnée et DataMining.
\item[PIB, GDP : ] Respectivement Produit Intérieur Brut et Gross Domestic
Product
\item[RNB, GNI : ] Respectivement Revenu National Brut et Gross National
Income
\item[UE : ] Union Européenne
\end{description}
