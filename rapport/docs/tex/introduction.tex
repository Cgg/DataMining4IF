\section{Introduction}

QQOQCP\\

Ce rapport présente le travail que nous avons effectué au cours du projet
de fouille de données en 4IF.


\subsection{Rappel du projet}

Le projet de fouille de donnée consiste à extraire des informations d'un
set de données contenant un ensemble de 48 attributs portant sur 209 pays,
mesurés en 2007. Il faut choisir soigneusement le ou les axes d'études en
restreignant l'ensemble des pays et/ou des attributs sur lesquels la
fouille porte.\\
L'étude est ensuite conduite en utilisant l'outil Open-Source Knime, avec
le plugin Weka.


\subsection{Axes d'études retenus}

\subsubsection{Ensemble de pays}
Afin de disposer d'un maximum d'attributs bien renseignés, et pour ne pas
se "perdre" au cours de l'étude, il nous faut diminuer le nombre de pays
étudiés.\\
Nous avons choisi de restreindre le champ de cette étude aux pays européen.
Ce choix nous permet de disposer d'un maximum d'informations et d'une
meilleure connaissance "apriori" du champ étudié.\\
Plus précisément, on distingue les pays membres de l'Union Européenne et
les pays externes, candidats (Turquie, par exemple) ou non (notamment la
Suisse ou la Norvège, mais également les cité-états telles que Andorre).\\

\subsubsection{Problématiques explorées}
En étudiant ce set de données, on souhaite explorer les questions suivantes
:
\begin{enumerate}
\item L'UE est-elle un ensemble de pays homogène ou bien peut-on déceler
des disparités entre les pays ?
\item Comment se situent les pays externes à l'UE par rapport à cette
dernière ?
\item Plus particulièrement, les pays candidats sont-ils recevables en se
basant sur les attributs étudiés ?
\end{enumerate}

\subsubsection{Ensemble d'attributs}
Certains des 48 attributs peuvent également être écartés car inutiles dans
l'étude que l'on souhaite conduire.\\
Pour le moment, nous souhaitons garder l'ensemble des attributs.


\subsection{Glossaire}

On présente les abréviations et termes de jargons utilisés dans ce rapport
:

\begin{description}
\item[FdD,DM : ] Respectivement Fouille de Donnée et DataMining.
\item[UE : ] Union Européenne
\end{description}
