\FloatBarrier
\vfill
\pagebreak
\section{Annexes}
\subsection*{Annexe A : Liste des États retenus pour l'étude}

On rappelle que les états européens constituent l'objet de notre étude. On
distingue trois catégories d'États : 
\begin{itemize}
\item États membres de l'UE
\item États candidats à l'adhésion à l'UE
\item États non-membres et non-candidat
\end{itemize}

\vskip 6pt

La catégorie des États candidats regroupent aussi bien les candidats dont
la candidature est actuellement examinée (Turquie, Islande par exemple) que
les États dont la candidature est encore en attente de traitement (Albanie,
Serbie notamment).

\begin{description}
\item[Etats membres de l'UE]\hfill
    \begin{itemize}
    \item Allemagne (RFA)
    \item Belgique
    \item France
    \item Italie
    \item Luxembourg
    \item Pays-Bas
    \item Irlande
    \item Royaume-Uni
    \item Danemark
    \item Grèce
    \item Espagne
    \item Portugal
    \item Autriche
    \item Finlande
    \item Suède
    \item Estonie
    \item Lettonie
    \item Lituanie
    \item Pologne
    \item République tchèque
    \item Slovaquie
    \item Hongrie
    \item Slovénie
    \item Chypres
    \item Malte
    \end{itemize}
    \vskip 6pt

\item[Etats candidats à l'adhésion à l'UE (reconnus et en attente)]\hfill
    \begin{itemize}
    \item Bulgarie
    \item Roumanie
    \item Turquie
    \item Croatie
    \item Macédoine
    \item Islande
    \item Monténégro
    \end{itemize}
    \vskip 6pt

\item[Etats non membre, non candidats à l'adhésion]\hfill
    \begin{itemize}
    \item Albanie
    \item Serbie
    \item Bosnie-Herzégovine
    \item Kosovo
	\item Suisse
	\item Norvège
	\item Arménie
	\item Biélorussie
	\item Moldavie
	\item Ukraine
    \end{itemize}
    \vskip 6pt

\end{description}
