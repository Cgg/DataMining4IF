\section{Vagues d'adhésions à l'Union Européenne}

L'UE s'est construite progressivement, les états la rejoignant par
"vagues". Dans cette étude, on utilisera ces appartenances aux différentes
vagues en les confrontant aux groupes d'états dégagés sur tel ou tel
critère : les groupes sont-ils similaires aux vagues ou non ?\\

De plus, on distinguera également les états ayant officiellement fait
candidature, les états souhaitant faire candidature, et les états
pressentis pour candidater à l'entrée dans l'UE.

\subsection{Etats membres}

\begin{description}
\item[Etats fondateurs, vague 0 (1957)]\hfill\\
    \begin{itemize}
    \item Allemagne (RFA)
    \item Belgique
    \item France
    \item Italie
    \item Luxembourg
    \item Pays-Bas
    \end{itemize}

\item[Vague 1 (1973)]\hfill\\
    \begin{itemize}
    \item Irlande
    \item Royaume-Uni
    \item Danemark
    \end{itemize}

\item[Vague 2 (1981)]\hfill\\
    \begin{itemize}
    \item Grèce
    \end{itemize}
    
\item[Vague 3 (1986)]\hfill\\
    \begin{itemize}
    \item Espagne
    \item Portugal
    \end{itemize}
    
\item[Vague 4 (1995)]\hfill\\ 
    \begin{itemize}
    \item Autriche
    \item Finlande
    \item Suède
    \end{itemize}

\item[Vague 5 (2004)]\hfill\\
    \begin{itemize}
    \item Estonie
    \item Lettonie
    \item Lituanie
    \item Pologne
    \item République tchèque
    \item Slovaquie
    \item Hongrie
    \item Slovénie
    \item Chypres
    \item Malte
    \end{itemize}

\item[Vague 6 (2007)]
    \begin{itemize}
    \item Bulgarie
    \item Roumanie
    \end{itemize}
\end{description}

\subsection{Etats officiellement candidats}

\begin{itemize}
\item Turquie
\item Croatie
\item Macédoine
\item Islande
\item Monténégro
\end{itemize}

\subsection{Etats potentiellement candidats}

\begin{itemize}
\item Albanie
\item Serbie
\item Bosnie-Herzégovine
\item Kosovo
\end{itemize}

