\section{Préparation des données}

Cette phase a pour but l'obtention de données permettant d'appliquer
facilement les phases suivantes. Elle comprend plusieurs étapes. 


\subsection{Réduction du nombre de pays étudiés}

Cela permet d'étudier un ensemble cohérent de pays (les pays européens dans
notre cas) ; en faisant cela on peut également écarter d'emblée les pays
dont aucun attributs (ou presque) n'est renseigné : par exemple les
cités-états européennes.\\

Dans notre cas, on se limite donc aux pays européens, en enlevant Andorre,
Monaco et le Liechstenstein. La liste complète est visible en annexe A.


\subsection{Réduction de la dimension du jeu de données}

On choisit avec soin les attributs sur lesquels on travaillera par la
suite. Ces attributs sont fonctions de l'axe d'étude choisi et seront
explicités à chaque fois.


\subsection{Ajout d'un attribut : appartenance à l'Union Europénne}

Afin d'évaluer l'influence d'une appartenance à l'UE dans les différents
domaines étudiés, on souhaite savoir si les pays du jeu de données en sont
membres ou non. À cette fin, on a choisit de simplement rajouter une
colonne "Membre de l'UE", comprenant un entier, associé à une forme dans
les graphismes présentés dans ce rapport. Les correspondances sont les
suivantes :

\begin{table}[h]
\centering
\caption{Attribut "Membre de l'UE"}
\begin{tabular}{c|c|c}
Code & Symbole & état de l'État \\ \hline
1 & $\bigtriangleup$ & État membre \\ \hline
0 & + & État candidat à l'adhésion \\ \hline
-1 & $\times$ & État non candidat, non membre
\end{tabular}
\end{table}
