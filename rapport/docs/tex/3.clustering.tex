\section{Modélisation}

\subsection{Clustering}

\subsubsection{Workflow}

Le pipeline de travail permet de clusteriser le jeu de données suivant
différents axes, que l'on peut sélectionner facilement en filtrant les
colonnes d'attributs en début de chaîne. Les attributs retenus sont
normalisés sur l'intervalle $[ 0 ; 1 ]$.

Ici un screenshot du pipeline\\

On se base sur la méthode "Fuzzy-C-means" afin de diviser le jeu de
données. Le clustering hiérarchique permet de déterminer le nombre de
clusters à rechercher et le composant "NOM ICI" permet de vérifier la stabilité
d'un tel clustering en l'effectuant un grand nombre de fois et en
permettant de comparer le nombre de clusters trouvés à chaque fois.\\

Ici un screen de la loop\\

Les deux figures suivantes montrent un clustering stable et instable sur
100 itérations. On voit que pour trois cluster recherchés, on trouvera
toujours trois cluster (clustering stable). En revanche, en recherchant
quatre cluster, le nombre de clusters trouvés varie grandement entre les
différentes itérations : le clustering est instable.

Ici les screens stable et instable, cote a cote\\


\subsubsection{Clustering basé sur les secteurs d'activité}

\subsubsection{Clustering basé sur la santé publique}

\subsubsection{Clustering basé sur la consommation et la production de
Hautes Technologies}

\subsection{Classification}

To be continued...
